\subsection{Command-line Interface}

A command-line interface was implemented to allow the user to interact with the system.
A terminal driver was developped to handle user input and execute commands and manages a 
buffer for multi-character commands. A command in this context is represented as a data 
structure called \code{Command}, which encapsulates the command type and an optional 
floating-point argument. The 'Command' struct contains:

\begin{itemize}
    \item the type of command issued by the user (e.g., \code{help}, \code{set_voltage}, \code{get_data}, \code{shutdown})
    \item an optional argument for commands that require additional information, such as a voltage value for \code{set_voltage}
\end{itemize}

This class provides methods for:
\begin{itemize}
    \item Handling input character by character.
    \item Parsing and executing user commands by interpreting the \code{Command} structure.
    \item Resetting and managing the internal command buffer used for building commands from user input.
\end{itemize}
