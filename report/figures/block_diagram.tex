\documentclass{standalone}
\usepackage{circuitikz}

\definecolor{darkgreen}{rgb}{0.0, 0.7, 0.0}

\begin{document}

\begin{circuitikz}
    \tikzstyle{every node}=[font=\small, align=center]

    % Main Board Square
    \draw[line width=2pt,fill=darkgreen ] (-2.25,-2) rectangle (11,12);
    \node at (-2.25,12) [anchor=north west] {\LARGE Main Board};

    % RP2040 Main Square
    \draw[line width=2pt,fill=lightgray ] (0,-1.5) rectangle (10,11);
    \node at (10,11) [anchor=north east] {\LARGE RP2040};
    \draw[line width=1pt,fill=white](1,-1) rectangle  node {\LARGE UART0} (3.5,0.5);
    \draw[line width=1pt,fill=white](6.5,-1) rectangle  node {\LARGE USB} (9.5,0.5);
    \draw[line width=1pt,fill=white](1,0.75) rectangle  node {\LARGE UART1} (3.5,2.25);
    \draw [ line width=1pt,fill=white ] (1,2.75) rectangle  node {\LARGE ADC3} (3.5,3.75);
    \draw [ line width=1pt,fill=white ] (1,4) rectangle  node {\LARGE ADC2} (3.5,5);
    \draw [ line width=1pt,fill=white ] (1,5.5) rectangle  node {\LARGE ADC1} (3.5,6.5);
    \draw [ line width=1pt,fill=white ] (1,6.75) rectangle  node {\LARGE I2C1} (3.5,8.25);
    \draw [ line width=1pt,fill=white ] (1,8.5) rectangle  node {\LARGE SPI0} (3.5,10.5);


    % Large red dotted rectangle
    \draw [red, dashed, thick] (-13,-2) rectangle (-2.75,5.25);
    \node at (-13,5.25) [red,anchor=north west] {\LARGE Sensors};

    % SD Card
    \draw [ line width=1pt,fill=white ] (-6,8.5) rectangle  node {\large SD Card} (-8.25,10.5);
    \node [font=\small, align=center, anchor=south west] at (-6,10.25) {CS};
    \node [font=\small, align=center, anchor=south west] at (-6,9.75) {MOSI};
    \node [font=\small, align=center, anchor=south west] at (-6,9.25) {CLK};
    \node [font=\small, align=center, anchor=south west] at (-6,8.75) {MISO};
    \draw [line width=1pt] (-6,10.25) to[short] (1,10.25);
    \draw [line width=1pt] (-6,9.75) to[short] (1,9.75);
    \draw [line width=1pt] (-6,9.25) to[short] (1,9.25);
    \draw [line width=1pt] (-6,8.75) to[short] (1,8.75);

    % DAC
    \draw [ line width=1pt,fill=white ] (-2,6.75) rectangle  node {\small DAC} (-1,8.25);
    \node [font=\small, align=center, anchor=south west] at (-1,7.75) {SCL};
    \node [font=\small, align=center, anchor=south west] at (-1,7.25) {SDA};
    \draw[ line width=1pt] (-1,7.75) to[short] (1,7.75); % DAC to IC21
    \draw[ line width=1pt] (-1,7.25) to[short] (1,7.25); % DAC to IC21


    % ADC and Load Cells
        % wires from ADC to Load Cells
        \draw [line width=1pt] (1,3.25) to[short] (-1,3.25);
        \draw [line width=1pt] (1,4.5) to[short] (-1,4.5);
        % add small op amps with the output connected to the end of the wires

        % Triangle for the first amplification at (-2, 3.5), flipped to point right
        \draw [line width=1pt,fill=white] (-2, 2.95) -- (-2, 3.55) -- (-1.5, 3.25) -- cycle;
        \draw [line width=1pt,fill=white] (-1.5,3.25) -- (-1,3.25); % Connect output to wire
        \node [anchor=south west] at (-1.8, 3.25) {In-Amp}; % Label for the second triangle

        % Triangle for the second amplification at (-2, 4.5), flipped to point right
        \draw [line width=1pt,fill=white] (-2, 4.2) -- (-2, 4.8) -- (-1.5, 4.5) -- cycle;
        \draw [line width=1pt,fill=white] (-1.5,4.5) -- (-1,4.5); % Connect output to wire
        \node [anchor=south west] at (-1.8, 4.5) {In-Amp}; % Label for the second triangle

        \draw [ line width=1pt,fill=white ] (-3,2.8125) rectangle  node {\large Loadcell 1} (-5.25,3.6875);
        \draw [ line width=1pt,fill=white ] (-3,4.0625) rectangle  node {\large Loadcell 2} (-5.25,4.9375);
        \draw [line width=1pt] (-3,4.5) to[short] (-2,4.5);
        \draw [line width=1pt] (-3,3.25) to[short] (-2,3.25);

        % Dew Point Generator
        \draw [ line width=1pt,fill=white ] (-6,5.5) rectangle  node {\large Dew Point Generator} (-11,8.25);
        \node [font=\small, align=center, anchor=south west] at (-6,7.375) {Control};
        \node [font=\small, align=center, anchor=south west] at (-6,6) {Data};
        \draw [line width=1pt] (-6,7.375) to[short] (-2,7.375);
        \draw [line width=1pt] (-6,6) to[short] (1,6);

        % Debugger
        \draw [ line width=1pt,fill=white ] (1,-5) rectangle  node {\large Debugger} (3.5,-4);
        \node [font=\small, align=center, anchor=south west, rotate=90] at (1.5,-4) {SWDIO};
        \node [font=\small, align=center, anchor=south west, rotate=90] at (2,-4) {SWCLK};
        \node [font=\small, align=center, anchor=south west, rotate=90] at (2.5,-4) {RX};
        \node [font=\small, align=center, anchor=south west, rotate=90] at (3,-4) {TX};
        \draw [line width=1pt] (1.5,-4) to[short] (1.5,-1);
        \draw [line width=1pt] (2,-4) to[short] (2,-1);
        \draw [line width=1pt] (2.5,-4) to[short] (2.5,-1);
        \draw [line width=1pt] (3,-4) to[short] (3,-1);

        % PC
        \draw [ line width=1pt,fill=white ] (6.5,-5) rectangle  node {\large PC} (9.5,-3.5);
        \node [font=\small, align=center, anchor=south west, rotate=90] at (7.5,-3.5) {Data +};
        \node [font=\small, align=center, anchor=south west, rotate=90] at (8.5,-3.5) {Data -};
        \draw [line width=1pt] (7.5,-3.5) to[short] (7.5,-1);
        \draw [line width=1pt] (8.5,-3.5) to[short] (8.5,-1);
















    % RS485 Transceiver
        \draw [ line width=1pt,fill=white ] (-2,0.5) rectangle  node {\small RS485 \\ Tran- \\ ciever} (-1,2.5);
        \node [font=\small, align=center, anchor=south west] at (-1,2) {RX};
        \node [font=\small, align=center, anchor=north west] at (-1,1) {TX};
        \draw[ line width=1pt] (-1,2) to[short] (1,2); % RS485 to UART0
        \draw[ line width=1pt] (-1,1) to[short] (1,1); % RS485 to UART0
    % SDI-12 Data lines
        \draw[ line width=1pt] (-2,2) to[short] (-11.5,2);
        \draw[ line width=1pt] (-2,1) to[short] (-11.5,1);
        \node [font=\small, align=center, anchor=south east] at (-3,1) {0-5V Data Line};
        % line A termination elements (text and hollow dot)
        \node [font=\LARGE] at (-12.75,2) {A};
        \draw [ line width=1pt](-11.5,2) to[short, -o] (-12.25,2) ;
        % line B termination elements (text and hollow dot)
        \node [font=\LARGE] at (-12.75,1) {B};
        \draw [ line width=1pt](-11.5,1) to[short, -o] (-12.25,1) ;
    % SDI-12 Sensor 1
        \node at (-4,1) [circ] {}; % sensor 1 node B connection
        \draw [ line width=1pt](-4,1) to[short] (-4,0.25); % wire to sensor 1
        \draw [ line width=1pt ] (-5,-1.75) rectangle  node {\large SDI-12 \\ Sensor 1} (-3,0.25);
    % SDI-12 Sensor 2
        \node at (-7,1) [circ] {}; % sensor 2 node B connection 
        \draw [ line width=1pt](-7,1) to[short] (-7,0.25); % wire to sensor 2
        \draw [ line width=1pt ] (-8,-1.75) rectangle  node {\large SDI-12 \\ Sensor 2} (-6,0.25);
    % SDI-12 Sensor 3
        \node at (-10.75,1) [circ] {}; % sensor N node B connection
        \draw [ line width=1pt](-10.75,1) to[short] (-10.75,0.25); % wire to sensor N
        \draw [ line width=1pt ] (-11.75,-1.75) rectangle  node {\large SDI-12 \\ Sensor N} (-9.75,0.25);
    
    \node [font=\Large] at (-8.9,-0.75) {$\cdots$}; % show that many sensors can be connected
    
    
\end{circuitikz}

\end{document}
