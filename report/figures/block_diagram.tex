\documentclass{standalone}
\usepackage{circuitikz}

\begin{document}

\begin{circuitikz}
    \tikzstyle{every node}=[font=\small, align=center]

    % Main Board Square
    \draw[line width=2pt,fill=lime ] (-2.25,-2) rectangle (11,12);
    \node at (-2.25,12) [anchor=north west] {\LARGE Main Board};

    % RP2040 Main Square
    \draw[line width=2pt,fill=lightgray ] (0,-1.5) rectangle (10,11);
    \draw[line width=1pt,fill=white](1,-1) rectangle  node {\LARGE UART0} (3.25,0.5);
    \draw[line width=1pt,fill=white](4.25,-1) rectangle  node {\LARGE USB} (5,0.5);
    \draw[line width=1pt,fill=white](1,0.75) rectangle  node {\LARGE UART1} (3.25,2.25);
    \draw [ line width=1pt,fill=white ] (1,3) rectangle  node {\LARGE ADC3} (3.25,3.875);
    \draw [ line width=1pt,fill=white ] (1,4.125) rectangle  node {\LARGE ADC2} (3.25,5);
    \draw [ line width=1pt,fill=white ] (1,5.5) rectangle  node {\LARGE ADC1} (3.25,6.25);
    \draw [ line width=1pt,fill=white ] (1,6.5) rectangle  node {\LARGE I2C1} (3.25,8.25);
    \draw [ line width=1pt,fill=white ] (1,8.5) rectangle  node {\LARGE SPI0} (3.25,10.5);


    % Large red dotted rectangle
    \draw [red, dashed, thick] (-13,-2) rectangle (-2.75,5.25);
    \node at (-13,5.25) [red,anchor=north west] {\LARGE Sensors};

    % SD Card
    \draw [ line width=1pt,fill=white ] (-3,8.5) rectangle  node {\large SD Card} (-5.25,10.5);
    \node [font=\small, align=center, anchor=south west] at (-2,10.25) {CS};
    \node [font=\small, align=center, anchor=south west] at (-2,9.75) {MOSI};
    \node [font=\small, align=center, anchor=south west] at (-2,9.25) {CLK};
    \node [font=\small, align=center, anchor=south west] at (-2,8.75) {MISO};
    \draw [line width=1pt] (-3,10.25) to[short] (1,10.25);
    \draw [line width=1pt] (-3,9.75) to[short] (1,9.75);
    \draw [line width=1pt] (-3,9.25) to[short] (1,9.25);
    \draw [line width=1pt] (-3,8.75) to[short] (1,8.75);

    % DAC
    \draw [ line width=1pt,fill=white ] (-2,5.5) rectangle  node {\small DAC} (-1,7.5);
    \node [font=\small, align=center, anchor=south west] at (-1,7.75) {SCL};
    \node [font=\small, align=center, anchor=south west] at (-1,7) {SDA};
    \draw[ line width=1pt] (-1,7.75) to[short] (1,7.75); % DAC to IC21
    \draw[ line width=1pt] (-1,7) to[short] (1,7); % DAC to IC21


    % ADC and Load Cells
        % wires from ADC to Load Cells
        \draw [line width=1pt] (1,3.5) to[short] (-1,3.5);
        \draw [line width=1pt] (1,4.5) to[short] (-1,4.5);
        % add small op amps with the output connected to the end of the wires

        % Triangle for the first amplification at (-2, 3.5), flipped to point right
        \draw [line width=1pt,fill=white] (-2, 3.2) -- (-2, 3.8) -- (-1.5, 3.5) -- cycle;
        \draw [line width=1pt,fill=white] (-1.5,3.5) -- (-1,3.5); % Connect output to wire
        \node [anchor=south west] at (-1.8, 3.5) {In-Amp}; % Label for the second triangle

        % Triangle for the second amplification at (-2, 4.5), flipped to point right
        \draw [line width=1pt,fill=white] (-2, 4.2) -- (-2, 4.8) -- (-1.5, 4.5) -- cycle;
        \draw [line width=1pt,fill=white] (-1.5,4.5) -- (-1,4.5); % Connect output to wire
        \node [anchor=south west] at (-1.8, 4.5) {In-Amp}; % Label for the second triangle

        \draw [ line width=1pt,fill=white ] (-3,3) rectangle  node {\large Loadcell 1} (-5.25,3.875);
        \draw [ line width=1pt,fill=white ] (-3,4.125) rectangle  node {\large Loadcell 2} (-5.25,5);
        \draw [line width=1pt] (-3,4.5) to[short] (-2,4.5);
        \draw [line width=1pt] (-3,3.5) to[short] (-2,3.5);

        % Dew Point Generator Data
        \draw [ line width=1pt,fill=white ] (-3,2) rectangle  node {\large Dew Point Generator Data} (-9.25,2.875);
        \draw [line width=1pt] (-3,2.5) to[short] (1,2.5);















    % RS485 Transceiver
        \draw [ line width=1pt,fill=white ] (-2,0.5) rectangle  node {\small RS485 \\ Tran- \\ ciever} (-1,2.5);
        \node [font=\small, align=center, anchor=south west] at (-1,2) {RX};
        \node [font=\small, align=center, anchor=north west] at (-1,1) {TX};
        \draw[ line width=1pt] (-1,2) to[short] (1,2); % RS485 to UART0
        \draw[ line width=1pt] (-1,1) to[short] (1,1); % RS485 to UART0
    % SDI-12 Data lines
        \draw[ line width=1pt] (-2,2) to[short] (-11.5,2);
        \draw[ line width=1pt] (-2,1) to[short] (-11.5,1);
        \node [font=\small, align=center, anchor=south east] at (-2,1) {0-5V Data Line};
        % line A termination elements (text and hollow dot)
        \node [font=\LARGE] at (-12.75,2) {A};
        \draw [ line width=1pt](-11.5,2) to[short, -o] (-12.25,2) ;
        % line B termination elements (text and hollow dot)
        \node [font=\LARGE] at (-12.75,1) {B};
        \draw [ line width=1pt](-11.5,1) to[short, -o] (-12.25,1) ;
    % SDI-12 Sensor 1
        \node at (-4,1) [circ] {}; % sensor 1 node B connection
        \draw [ line width=1pt](-4,1) to[short] (-4,0.25); % wire to sensor 1
        \draw [ line width=1pt ] (-5,-1.75) rectangle  node {\large SDI-12 \\ Sensor 1} (-3,0.25);
    % SDI-12 Sensor 2
        \node at (-7,1) [circ] {}; % sensor 2 node B connection 
        \draw [ line width=1pt](-7,1) to[short] (-7,0.25); % wire to sensor 2
        \draw [ line width=1pt ] (-8,-1.75) rectangle  node {\large SDI-12 \\ Sensor 2} (-6,0.25);
    % SDI-12 Sensor 3
        \node at (-10.75,1) [circ] {}; % sensor N node B connection
        \draw [ line width=1pt](-10.75,1) to[short] (-10.75,0.25); % wire to sensor N
        \draw [ line width=1pt ] (-11.75,-1.75) rectangle  node {\large SDI-12 \\ Sensor N} (-9.75,0.25);
    
    \node [font=\Large] at (-8.9,-0.75) {$\cdots$}; % show that many sensors can be connected
    
    
\end{circuitikz}

\end{document}
