\section{Discussion}

The project successfully met the primary objectives outlined in the introduction, offering a reliable and efficient solution for monitoring and controlling environmental parameters in tropical plant research. By integrating the RP2040 microcontroller with SDI-12 environmental sensors, a load cell, and a smart interface for the dew point generator, the system enables accurate data collection and environmental control. The ability to control the dew point generator with voltage outputs accurate to within $\pm 0.02,\mathrm{V}$, receive data from multiple SDI-12 sensors, log data automatically to an SD card, and interact via a command-line interface demonstrates the system's effectiveness in fulfilling the researcher's requirements.

The functionality is consolidated into a single package costing approximately 10 dollars, making it a cost-effective alternative to commercial solutions like those from Campbell Scientific. The open-source nature of the project enhances accessibility and allows for customization, aligning with the goal of providing a flexible tool for environmental research.

However, the system has limitations that affect its commercial viability. The hardware issue involving swapped SDA and SCL pins on the PCB highlights the need for meticulous design reviews to prevent such errors. While the command-line interface is suitable for the specific researcher accustomed to similar systems, it may not be user-friendly for a broader audience. Implementing a graphical user interface (GUI) could improve usability for other potential users.

The use of a 10-bit DAC provides acceptable control over the dew point generator, but upgrading to a 12-bit DAC would enhance voltage output precision, further improving the system's performance in controlling environmental conditions. Additionally, while the hardware is designed for extensibility with numerous screw terminals for future expansions, the software lacks the necessary features to fully realize this potential. Implementing dynamic sensor discovery and more flexible communication protocols would enhance the system's scalability and adaptability.

In terms of design choices, the decision to use RS485 transceivers for SDI-12 communication proved effective in handling inverted logic levels and enhancing noise immunity. However, the reliance on hardcoded sensor addresses limits scalability and requires prior knowledge of sensor configurations. The use of an instrumentation amplifier improved load cell readings, but mechanical vibrations and environmental noise still impacted data stability, indicating a need for better signal conditioning.

Overall, the project demonstrates a successful proof of concept that meets the immediate needs of the researcher but requires further refinement to enhance reliability, scalability, and user-friendliness. Addressing these areas will be crucial for transitioning the system from a prototype to a more widely adoptable solution in environmental research applications.
