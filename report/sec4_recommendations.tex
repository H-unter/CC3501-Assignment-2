\section{Recommendations}

Based on the design process and challenges encountered, several recommendations can be made for future engineers who might revise this work:

1. Implement Dynamic Sensor Discovery: To enhance the system's scalability and flexibility, incorporate the SDI-12 "Address Query" command (?!) to dynamically discover all connected sensors. This would eliminate the need for hardcoded sensor addresses and allow the system to support any number of SDI-12 sensors without software modifications.

2. Develop Robust Error Handling: Improve the software to handle communication errors, sensor timeouts, and unexpected responses. Implementing retries, acknowledgments, and exception handling will make the system more reliable in field conditions.

3. Optimise Hardware Design for Extensibility: Ensure that hardware interfaces, such as screw terminals and connectors, are designed to accommodate future expansions. Consider modular designs that allow for easy addition or replacement of components.

4. Thorough PCB Layout Review: Implement a rigorous design review process before finalising the PCB layout. This includes checking pin assignments, signal integrity, and component placements to prevent issues like swapped pins or incorrect resistor sizing.

5. Enhanced Signal Conditioning: For analogue inputs like the load cell, consider implementing hardware filtering and shielding to reduce noise and improve signal stability. This might include adding low-pass filters or using differential signal processing techniques.

6. Documentation and Standardisation: Maintain comprehensive documentation of both hardware and software components. Adhering to coding standards and providing clear comments will facilitate future development and maintenance.

7. Power Management: Explore power-saving features and strategies, especially if the system is to be deployed in remote locations where power availability is limited.

By addressing these areas, future iterations of the system can achieve greater robustness, scalability, and ease of use, ultimately enhancing its value for environmental research applications.