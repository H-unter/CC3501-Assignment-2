\subsection{Recommendations}

Based on the design process and challenges encountered, several recommendations can be made for future engineers who might revise this work:

\textbf{1. Implement Dynamic Sensor Discovery}: Incorporate the SDI-12 "Address Query" command (?!) to dynamically discover all connected sensors. This would eliminate the need for hardcoded sensor addresses and allow the system to support any number of SDI-12 sensors without software modifications.

\textbf{2. Develop a User-Friendly Interface}: Implement a graphical user interface (GUI) to make the system more accessible to users beyond the initial researcher, enhancing usability and broadening its appeal.

\textbf{3. Develop Robust Error Handling}: Improve the software to handle communication errors, sensor timeouts, and unexpected responses. Implementing retries, acknowledgments, and exception handling will make the system more reliable in field conditions.

\textbf{4. Implement Efficient Task Management}: Introduce an SDI-12 command queue or utilize a real-time operating system (RTOS) to manage sensor requests. This would ensure smoother task allocation, prevent conflicts, and improve response handling for both fast and slow sensors.

\textbf{5. Thorough PCB Layout Review}: Implement a rigorous design review process before finalizing the PCB layout to prevent hardware errors like swapped SDA and SCL pins or incorrect resistor sizing.

\textbf{6. Enhanced Signal Conditioning}: For analogue inputs like the load cell, consider implementing hardware filtering and shielding to reduce noise and improve signal stability. This might include adding low-pass filters or using differential signal processing techniques.

\textbf{7. Upgrade DAC Precision}: Replace the current 10-bit DAC with a 12-bit DAC to improve voltage output accuracy for the dew point generator.

By addressing these areas, future iterations of the system can achieve greater robustness, scalability, and ease of use, ultimately enhancing its value for environmental research applications.