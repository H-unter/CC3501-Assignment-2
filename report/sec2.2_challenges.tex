\section{Technical Challenges and Solutions}

\subsection{SDI-12 Sensor Communication Issues}
During the testing of the SDI-12 sensors, initial attempts to communicate using RS485 transceivers failed due to improper framing of byte streams. The strict timing requirements of the SDI-12 protocol made it difficult to achieve reliable communication using standard UART settings. This issue was addressed by applying the \code{uart_set_format()} function, which allowed the RP2040 to handle the SDI-12 protocol's timing more accurately. Additionally, a \code{custom is_timed_out()} function was created to automate character reception without using traditional sleep methods, ensuring precise byte timing and avoiding data loss during transmission.

\subsection{Load Cell Signal Scaling and Stability}
The MT-603 load cell exhibited a significant dead zone and provided inconsistent readings due to mechanical vibrations in the experimental setup. Software-based noise reduction techniques, such as averaging past data points, were implemented to mitigate the effect of oscillations. This software solution was preferred over hardware filtering due to the constraints of the existing apparatus.

\note{Issues with clk and data lines being connected the wrong way around on the RP2040}