\section{Intro}

This report details the design and implementation of an embedded system functioning as both a scientific data logger and a smart interface with an analogue dew point generator. Utilizing the RP2040 microcontroller, SDI-12 environmental sensors, and a load cell, the system collects and processes data for climate modelling. The project aims to offer a reliable and efficient solution for monitoring and controlling environmental parameters, especially in researching tropical plant behavior under varying climate conditions.

The LI-610 Dew Point Generator is a precision instrument that produces a stable gas stream with a controlled dew point \cite{LICORDocument}. It employs Peltier thermoelectric coolers to regulate water reservoir temperatures, ensuring the air stream is fully saturated with water vaporm \cite{LICORDocument}. This precise dew point control is vital in environmental research, preventing condensation in climate-controlled chambers and maintaining experimental conditions and data accuracy.

Accurate, continuous monitoring of environmental parameters is crucial for tropical plant research, but traditional methods are labor-intensive, error-prone, and physical presence in climate-controlled rooms can disrupt experiments. Commercial solutions like Campbell Scientific are often expensive and closed-source, limiting accessibility and customization for researchers. This project provides an open-source, cost-effective alternative for remote monitoring and control. By integrating various sensors and a smart interface for the dew point generator, the system enables the simulation of different climate conditions and monitoring of plant responses without entering the controlled environment, ensuring data integrity while enhancing flexibility and affordability.