\section{Intro}

This report documents the design and implementation of an embedded system developed to serve as both a scientific data logger and a smart interface with an analogue dew point generator. The system utilizes the RP2040 microcontroller and various sensors, including SDI-12 environmental sensors and a load cell, to collect and process data for use in climate modelling applications. The project aims to provide a reliable and efficient solution for monitoring and controlling environmental parameters, particularly in the context of research on tropical plant behaviour under varying climate conditions.

The LI-610 Dew Point Generator is a precision instrument designed to produce a stable stream of gas with a precisely controlled dew point. It uses Peltier thermoelectric coolers to regulate the temperature of water reservoirs, ensuring the air stream is fully saturated with water vapor. This precise control of dew point is critical in environmental research, as it helps prevent condensation inside climate-controlled chambers, which could disrupt experimental conditions and data accuracy.

Accurate and continuous monitoring of environmental parameters is essential for tropical plant research, but traditional data collection methods are labor-intensive and prone to error. Moreover, physical presence in climate-controlled rooms can disturb experimental conditions. Commercial solutions like those from Campbell Scientific are often expensive and use closed-source technology, limiting accessibility and customization options for researchers. This project addresses these limitations by providing an open-source, cost-effective alternative that enables remote monitoring and control of environmental parameters. By integrating various sensors and offering a smart control interface for the dew point generator, the system allows researchers to simulate different climate conditions and monitor plant responses without entering the controlled environment, ensuring data integrity while maintaining flexibility and affordability.

\note{Digital to analogue conversion and PWM Background (only a couple of sentences) Big Q did stuff on this}